\section{Eigenwerte und Eigenvektoren}


% Grundüberlegung (Ausgangsmatrix muss quadratisch sein)
\begin{frame}
    \frametitle{Grundüberlegung zu Eigenwerten und Vektoren}
    \begin{itemize}
        \item Ausgangsmatrix A muss quadratisch sein
        \item Wir suchen einen Vektor $\vec{u}$ der mit einer Skalarzahl $\lambda$ multipliziert gleich sich selbst multipliziert mit der Ausgangsmatrix $A$ ist
    \end{itemize}
    $\rightarrow A \cdot \vec{u} = \lambda \cdot \vec{u}$
\end{frame}
% gegenüberstellung Au = lambda u
% umformung zu (A - lambda I)u = 0 (grund dafür) -> dies ist da charakteristisches polynom chi
\begin{frame}
    \frametitle{Berechnung der Eigenwerte}
    \begin{itemize}
        \item Umformen von $A \cdot \vec{u} = \lambda \cdot \vec{u}$, um Lambda berechnen zu können
        \item $(A- \lambda I)\vec{u} = \vec{0}$
        \item Deshalb muss $Ker(A - \lambda I) \neq \vec{0}$ sein
        \item $\rightarrow det(A - \lambda I) = 0$ (Charakteristisches Polynom Chi)
        \item Mit dem charakteristischen Polynom Chi können wir die Eigenwerte berechnen
    \end{itemize}
\end{frame}
% Nullstellen von det(A- lambda I) = 0 sind eigenwerte
\begin{frame}
    \frametitle{Berechnung der Eigenwerte}
    \begin{itemize}
        \item Aufstellen des Polynoms zur Berechnung der Determinante
        \item Nullstellen dieses Polynoms sind die Eigenwerte
    \end{itemize}
    Beispiel:
    $det(\begin{pmatrix}
        1 & 1 \\
        1 & 1 \\
    \end{pmatrix} - \lambda \cdot \begin{pmatrix}
        1 & 0 \\
        0 & 1 \\
    \end{pmatrix}) = det(\begin{pmatrix}
        1- \lambda & 1 \\
        1 & 1- \lambda \\
    \end{pmatrix}) = 0$

    $(1- \lambda)^2 - 1 = \lambda ^2 - 2 \lambda = 0 $
    \newline
    
    $\lambda _{1} = 0 $
    \newline
    $\lambda _{2} = 2$

\end{frame}
% Bei dreiecksmatrix sind die Eigenwerte die diagonalwerte

% gibt es doppelte Nullstellen hat dieser Eigenwert beispielsweise die Vielfachheit 2
% Summe aller eigenwerte (inklusive Vielfachheit) = det(A) = Summe aller diagonalwerte von A
\begin{frame}
    \frametitle{Eigenschaften der Eigenwerte}
    \begin{itemize}
        \item Bei Dreiecksmatitzen sind die Eigenwerte die Diagonalwerte
        \item gibt es doppelte Nullstellen hat dieser Eigenwert beispielsweise die Vielfachheit 2
        \item  Summe aller eigenwerte jeweils multipliziert mit ihrer Vielfachheit $= det(A) =$ Summe aller diagonalwerte von A
        \item In unserem Beispiel, beide Eigenwerte mit Vielfachheit 1 $\rightarrow 1+1 = 2$
        \item Eine Matrix ist regulär wenn alle Eigenwerte ungleich 0 sind
    \end{itemize}
\end{frame}
% Jeder eigenwert hat eigene Eigenvektoren
% Berechnung der Eigenvektoren
\begin{frame}
    \frametitle{Berechnung der Eigenvektoren}
    \begin{itemize}
        \item Zu jedem Eigenwert gibt es eigene Eigenvektoren
        \item Diese sind zu denen der anderen linear unabhängig
    \end{itemize}
    Eigenvektor zu $\lambda _{i}$ ist Lösung der Gleichung $(A - \lambda _{i} I) \vec{u} = \vec{0}$
    $(\begin{pmatrix}
        1 & 1 \\
        1 & 1 \\
    \end{pmatrix} - 0 \cdot \begin{pmatrix}
        1 & 0 \\
        0 & 1 \\
    \end{pmatrix}) \cdot \begin{pmatrix}
        u _{1} \\
        u _{2} \\
    \end{pmatrix} = \vec{0}$
    \newline
    Eigenvektor zu $\lambda _{1}$ ist Lösung der Gleichung, dies ist für alle Eigenvektoren zu wiederholen
\end{frame}

\begin{frame}
    \frametitle{Eigenschaften von Eigenvektoren}
    \begin{itemize}
        \item transponierte Matrix $A ^{T}$ hat selbe Eigenwerte wie A
        \item Ist die Matrix transponierbar sind die Eigenvektoren verschiedener Eigenwerte orthogonal
    \end{itemize}
\end{frame}

% Übungen
\begin{frame}
    \frametitle{Übungen}
    Bestimmt die Eigenwerte und Eigenvektoren der folgenden Matrizen. Sind die Matrizen regulär?
    \newline
    $A = \begin{pmatrix}
        3 & 1 \\
        1 & 3 \\
    \end{pmatrix}$
    \newline
    $B = \begin{pmatrix}
        -8 & -2 \\
        16 & 4 \\
    \end{pmatrix}$
\end{frame}

\begin{frame}
    \frametitle{Lösungen}
    Zu 1.: $\lambda_1 = 4$ und $\lambda_2 = 2$ Daher regulär
    \newline
    Eigenvektor zu $\lambda_1$ ist $\vec{u_1} = \begin{pmatrix}
        1 \\
        1 \\
    \end{pmatrix}$
    \newline
    Eigenvektor zu $\lambda_2$ ist $\vec{u_2} = \begin{pmatrix}
        1 \\
        -1 \\
    \end{pmatrix}$
    \vspace{0.3cm}
 \newline
    Zu 2. : $\lambda_1 = -4$ und $\lambda_2 = 0$ Daher nicht regulär
    \newline
    Eigenvektor zu $\lambda_1$ ist $\vec{u_1} = \begin{pmatrix}
        -1 \\
        2 \\
    \end{pmatrix}$
    \newline
    Eigenvektor zu $\lambda_2$ ist $\vec{u_2} = \begin{pmatrix}
        -1 \\
        4 \\
    \end{pmatrix}$
\end{frame}