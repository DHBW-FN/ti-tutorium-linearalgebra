\section{Rang Kern Bild}
\subsection{Rang}
\begin{frame}
    \frametitle{Rang}
    \begin{itemize}
        \item $rang(A) \Leftrightarrow$ ist die Anzahl linear unabhängiger Zeilen bzw. Spalten (Wegen $rang(A)=rang(A^\top)$)
        \item Elementare Zeilenumformungen verändern den Rang nicht $\Rightarrow$ in ZSF sind Zeilen mit führender 1 linear unabhängig
        \item Rang eines LGS = Rang der Koeffizientenmatrix
        \item LGS lösbar $\Leftrightarrow$ Rang der Koeffizientenmatrix = Rang der Erweiterten Matrix
    \end{itemize}
\end{frame}

\begin{frame}
	\frametitle{Beispiel}
	\begin{align*}
		A = \begin{pmatrix}
			2 & 4 & 0 \\
			0 & 0 & 1 \\
			0 & 0 & 0
		\end{pmatrix}
	\end{align*}
	\begin{itemize}
		\item $rang(A) = 2$
		\item $A$ ist schon fast in Zeilenstufenform und somit kann der Rang leicht abgelesen werden
	\end{itemize}
\end{frame}

\begin{frame}
	\frametitle{Übung I}
	Bestimme den Rang der Matrix
	\begin{align*}
		\begin{pmatrix}
			1 & -2 & 1 \\
			3 & -2 & -4 \\
			2 & 6 & -4 \\
			1 & 3 & -2 
		\end{pmatrix}
	\end{align*}
\end{frame}

\begin{frame}
	\frametitle{Übung I - Lösung}
	Bestimme den Rang der Matrix
	\begin{align*}
		\begin{pmatrix}
			1 & -2 & 1 \\
			3 & -2 & -4 \\
			2 & 6 & -4 \\
			1 & 3 & -2 
		\end{pmatrix} \\
		\leftrightarrow
		\begin{pmatrix}
			1 & -2 & 1 \\
			0 & 4 & -7 \\
			0 & 0 & \frac{23}{2} \\
			0 & 0 & 0
		\end{pmatrix} 
	\end{align*}
	$\rightarrow$ der Rang von $A$ ist somit $3$.
\end{frame}


\subsection{Bild}
\begin{frame}
    \frametitle{Bild}
    \begin{itemize}
        \item $Bild(A)={Ax | x\in \mathbb{R}^n}$
        \item Menge aller Elemente, die durch Anwendung der Abbildung $A$ auf den Vektorraum $\mathbb{R}^n$ entstehen
    \end{itemize}
\end{frame}

\subsection{Kern}
\begin{frame}
    \frametitle{Kern}
    \begin{itemize}
        \item $Ker(A)={x | Ax=0}$
        \item Menge aller Elemente, die auf 0 abgebildet werden
        \item $Ker(A)$ ist Lösungsmenge des zu A gehörenden homogenen LGS
    \end{itemize}
\end{frame}

\begin{frame}
	\frametitle{Beispiel I}
	\begin{align*}
		A = \begin{pmatrix}
			1 & 1 & 2 \\
			0 & 1 & 1 \\
			1 & 0 & 1 
		\end{pmatrix}
	\end{align*}
	Um den Kern zu bestimmen muss das homogene LGS Ax = 0 gelöst werden.
	\begin{align*}
		\begin{pmatrix}
			1 & 1 & 2 \\
			0 & 1 & 1 \\
			1 & 0 & 1 
		\end{pmatrix} \rightarrow
		\begin{pmatrix}
			1 & 1 & 2 \\
			0 & 1 & 1 \\
			0 & -1 & -1 
		\end{pmatrix} \rightarrow
		\begin{pmatrix}
			1 & 0 & 1 \\
			0 & 1 & 1 \\
			0 & 0 & 0 
		\end{pmatrix} 
	\end{align*}
\end{frame}

\begin{frame}
	\frametitle{Beispiel II}
	\begin{align*}
		a + c = 0 \\
		b + c = 0 \\
	\end{align*}
	Wir definieren dann zum Beispiel $c = t$ und erhalten
	\begin{align*}
		a = -t \\
		b = -t 
	\end{align*}
	Somit kommen wir dann auf Ker(A) = 
	$\{t \cdot (-1, -1, 1) | t \in \mathbb{R}\}$
\end{frame}

\begin{frame}
	\frametitle{Beispiel III}
	Da wir die Matrix A den Rang 2 hat haben wir 2 unabhängige Spaltenvektoren.
	Daraus folgt das Bild:
	\begin{align*}
		Bild(A) = \{k_1 \cdot \begin{pmatrix}
			1 \\
			0 \\
			1 
		\end{pmatrix} + k_2 \cdot \begin{pmatrix}
			1 \\
			1 \\
			0 
		\end{pmatrix} | k_1, k_2 \in \mathbb{R}\}
	\end{align*}
\end{frame}
