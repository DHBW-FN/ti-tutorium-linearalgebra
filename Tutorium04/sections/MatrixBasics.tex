\section{Reelle Matrizen}

\subsection{Basics}
\begin{frame}
    \frametitle{Was ist das?}
    \begin{itemize}
        \item Eine Matrix ist eine Anordnung von Zahlen in Zeilen und Spalten.
        \item Die Zahlen in einer Matrix werden als Elemente oder Koeffizienten bezeichnet.
        \item Eine $m \times n$ Matrix hat $m$ Zeilen und $n$ Spalten.
        \item Die Elemente einer Matrix werden mit $a_{ij}$ bezeichnet.
        \item Beispiel: $A = \begin{pmatrix}
                                 1 & 2 & 3 \\
                                 4 & 5 & 6 \\
                                 7 & 8 & 9
        \end{pmatrix}$
    \end{itemize}
\end{frame}

\begin{frame}
    \frametitle{Was ist das?}
    \begin{itemize}
        \item Zeilen einer Matrix werden auch als Zeilenvektoren bezeichnet.
        \item Spalten einer Matrix werden auch als Spaltenvektoren bezeichnet.
        \item Gilt $Anzahl Zeilen = Anzahl Spalten$ dann ist die Matrix quadratisch.
    \end{itemize}
\end{frame}

\subsection{Einfache Rechenoperationen}
\begin{frame}
    \frametitle{Einfache Rechenoperationen}
    $A + B =
    \begin{pmatrix}
        a_{11} & a_{12} & \dots  & a_{1n} \\
        a_{21} & a_{22} & \dots  & a_{2n} \\
        \vdots & \vdots & \ddots & \vdots \\
        a_{m1} & a_{m2} & \dots  & a_{mn}
    \end{pmatrix}
    +
    \begin{pmatrix}
        b_{11} & b_{12} & \dots  & b_{1n} \\
        b_{21} & b_{22} & \dots  & b_{2n} \\
        \vdots & \vdots & \ddots & \vdots \\
        b_{m1} & b_{m2} & \dots  & b_{mn}
    \end{pmatrix}
    =
    \begin{pmatrix}
        a_{11} + b_{11} & a_{12} + b_{12} & \dots  & a_{1n} + b_{1n} \\
        a_{21} + b_{21} & a_{22} + b_{22} & \dots  & a_{2n} + b_{2n} \\
        \vdots          & \vdots          & \ddots & \vdots          \\
        a_{m1} + b_{m1} & a_{m2} + b_{m2} & \dots  & a_{mn} + b_{mn}
    \end{pmatrix}$
\end{frame}

\begin{frame}
    \frametitle{Einfache Rechenoperationen}
    $k \cdot A =
    k \cdot
    \begin{pmatrix}
        a_{11} & a_{12} & \dots  & a_{1n} \\
        a_{21} & a_{22} & \dots  & a_{2n} \\
        \vdots & \vdots & \ddots & \vdots \\
        a_{m1} & a_{m2} & \dots  & a_{mn}
    \end{pmatrix}
    =
    \begin{pmatrix}
        k \cdot a_{11} & k \cdot a_{12} & \dots  & k \cdot a_{1n} \\
        k \cdot a_{21} & k \cdot a_{22} & \dots  & k \cdot a_{2n} \\
        \vdots         & \vdots         & \ddots & \vdots         \\
        k \cdot a_{m1} & k \cdot a_{m2} & \dots  & k \cdot a_{mn}
    \end{pmatrix}$
\end{frame}

\begin{frame}
    \frametitle{Rechenregeln}
    Es sei $A, B, C \in \mathbb{R}^{m \times n}$ und $k, h \in \mathbb{R}$.
    \begin{itemize}
        \item $A + B = B + A$ (Kommutativgesetz)
        \item $A + (B + C) = (A + B) + C$ (Assoziativgesetz)
        \item $A + 0 = A$ (Existenz des neutralen Elements)
        \item $A + (-A) = 0$ (Existenz des inversen Elements)
        \item $k(hA) = (kh)A$
        \item $1A = A$
        \item $k(A + B) = kA + kB$ (Distributivgesetz 1)
        \item $(k + h)A = kA + hA$ (Distributivgesetz 2)
    \end{itemize}
\end{frame}

\subsection{Besondere Matrizen}
\begin{frame}
    \frametitle{Besondere Matrizen}
    \begin{itemize}
        \item Nullmatrix $A = 0 = \begin{pmatrix}
                                    0 & 0 & \dots  & 0 \\
                                    0 & 0 & \dots  & 0 \\
                                    \vdots & \vdots & \ddots & \vdots \\
                                    0 & 0 & \dots  & 0
        \end{pmatrix}$
    \end{itemize}
\end{frame}
