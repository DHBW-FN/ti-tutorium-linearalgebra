\section{Matrix Transponieren}
\begin{frame}
\frametitle{Transponieren von Matrizen}
Man vertauscht die Zeilen und Spalten einer Matrix. Die Transponierte einer Matrix A ist die Matrix A\textsuperscript{T}.
Eine (m,n)-Matrix A hat die Transponierte A\textsuperscript{T} die Dimension (n,m). Die Transponierte einer Transponierten ist die Matrix selbst.
\newline

Beispiel:\\
$\begin{pmatrix}
1 & 2 & 3\\
4 & 5 & 6
\end{pmatrix}\textsuperscript{T} = \begin{pmatrix}
1 & 4\\
2 & 5\\
3 & 6
\end{pmatrix}$

\end{frame}

\subsection{Rechenregeln}
\begin{frame}
\frametitle{Transponierte von Matrizen - Rechenregeln}
Für das Transponieren gelten folgende Rechenregeln:
\begin{itemize}
\item $A\textsuperscript{T}\textsuperscript{T} = A$
\item $(A+B)\textsuperscript{T} = A\textsuperscript{T} + B\textsuperscript{T}$
\item $(AB)\textsuperscript{T} = B\textsuperscript{T}A\textsuperscript{T}$
\end{itemize}

Quadratische Matrix wo $A = A\textsuperscript{T}$ ist, nennt man \textbf{symmetrisch}.\\
Eine Matrix wo $A = -A\textsuperscript{T}$ ist, nennt man \textbf{antisymmetrisch} oder \textbf{schiefsymmetrisch}.
\end{frame}

\subsubsection{Übungen}
\begin{frame}
\frametitle{Transponieren von Matrizen - Übungen}
\begin{enumerate}
\item $A = \begin{pmatrix}
1 & 0 & 3\\
2 & -1 & -2\\
-3 & 0 & 4
\end{pmatrix}$
\item A\textsuperscript{T}+B\textsuperscript{T}
$A = \begin{pmatrix}
1 & 4 \\
5 & 6
\end{pmatrix}
B = \begin{pmatrix}
2 & 0\\
-2 & -3
\end{pmatrix}$
\end{enumerate}
\end{frame}


\subsubsection{Lösungen}
\begin{frame}
\begin{enumerate}
\item $A\textsuperscript{T} = \begin{pmatrix}
1 & 2 & -3\\
0 & -1 & 0\\
-3 & 0 & 4
\end{pmatrix}$
\item $A + B = \begin{pmatrix}
1 & 4\\
5 & 6
\end{pmatrix} + \begin{pmatrix}
2 & 0\\
-2 & -3
\end{pmatrix} = \begin{pmatrix}
3 & 4\\
3 & 3
\end{pmatrix} 
\newline
=> (A+B)\textsuperscript{T} = \begin{pmatrix}
3 & 3\\
4 & 3
\end{pmatrix}$
\end{enumerate}
\end{frame}

\subsection{Dreiecksmatrizen}
\begin{frame}
\frametitle{Obere - Dreiecksmatrix}
Eine Matrix A ist \textbf{obere Dreiecksmatrix} wenn alle Elemente unterhalb der Hauptdiagonale 0 sind.\\
$A = \begin{pmatrix}
a_{11} & a_{12} & a_{13} & \dots & a_{1n}\\
0 & a_{22} & a_{23} & \dots & a_{2n}\\
0 & 0 & a_{33} & \dots & a_{3n}\\
\vdots & \vdots & \vdots & \ddots & \vdots\\
0 & 0 & 0 & \dots & a_{nn}
\end{pmatrix}$
\end{frame}

\begin{frame}
\frametitle{Untere - Dreiecksmatrix}
Eine Matrix A ist \textbf{untere Dreiecksmatrix} wenn alle Elemente oberhalb der Hauptdiagonale 0 sind.\\
$A = \begin{pmatrix}
a_{11} & 0 & 0 & \dots & 0\\
a_{21} & a_{22} & 0 & \dots & 0\\
a_{31} & a_{32} & a_{33} & \dots & 0\\
\vdots & \vdots & \vdots & \ddots & \vdots\\
a_{n1} & a_{n2} & a_{n3} & \dots & a_{nn}
\end{pmatrix}$
\end{frame}

\subsection{Diagonalmatrix}
\begin{frame}
\frametitle{Diagonalmatrix}
Eine Matrix A ist \textbf{Diagonalmatrix} wenn alle Elemente außerhalb der Hauptdiagonale 0 sind.\\
$A = \begin{pmatrix}
a_{11} & 0 & 0 & \dots & 0\\
0 & a_{22} & 0 & \dots & 0\\
0 & 0 & a_{33} & \dots & 0\\
\vdots & \vdots & \vdots & \ddots & \vdots\\
0 & 0 & 0 & \dots & a_{nn}
\end{pmatrix}$
\end{frame}


\subsection{Einheitsmatrix}
\begin{frame}
\frametitle{Einheitsmatrix}
Eine Matrix A ist \textbf{Einheitsmatrix} wenn alle Elemente außerhalb der Hauptdiagonale 0 sind und die Elemente auf der Hauptdiagonale 1 sind.\\
$A = \begin{pmatrix}
1 & 0 & 0 & \dots & 0\\
0 & 1 & 0 & \dots & 0\\
0 & 0 & 1 & \dots & 0\\
\vdots & \vdots & \vdots & \ddots & \vdots\\
0 & 0 & 0 & \dots & 1
\end{pmatrix}$
Diese wird auch mit $\mathbb{I}$ bezeichnet.
\end{frame}
