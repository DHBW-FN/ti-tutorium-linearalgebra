\section{Lineare Abbildungen}
\begin{frame}
    \frametitle{Zweck}
    Lineare Abbildungen können als Transformationen von Vektoren verstanden werden. \\
    z.B. Drehung, Verschiebung, Spiegelung, Stauchung, Streckung, ...
\end{frame}

\subsection{Definition}

\begin{frame}
	\frametitle{Haupteigenschaft der linearen Abbildungen}
	Bei linearen Abbildungen wird immer eine Abbildungsmatrix F mit der Ausgangsmatrix A (oder Ausgangsvektor) multipliziert. $F(A)=F*A$
	\newline
	Dafür muss gelten:
	\begin{itemize}
		\item $F(A+B)= F*A+F*B = F(A)+F(B)$
		\item $F(k*A)= k*F*A = k*F(A)$
	\end{itemize}
\end{frame}

\subsection{Vorgehensweise}

\begin{frame}
	\frametitle{Grundverständnis}
	Abbildung durch Multiplikation einer Matrix mit Ausgangsvektor \\
	Hier im zweidimensionalen Raum eine Spiegelung an der x-Achse:
	\begin{itemize}
		\item x-Koordinate bleibt unverändert, y-Koordinate wird negiert
	  \end{itemize}
	  \begin{gather*}
		F = \begin{pmatrix}
			1 & 0 \\
			0 & -1
			\end{pmatrix} \\ \end{gather*}

		Zeilen der Matrix entsprechen den Bildern der Einheitsvektoren
		\begin{gather*} in X-Richtung:  \begin{pmatrix}
			1 \\
			0
			\end{pmatrix} \\
		in Y-Richtung: \begin{pmatrix}
			0 \\
			-1
			\end{pmatrix} \\ \end{gather*}
\end{frame}

\subsection{Transformationen}

\begin{frame}
	\frametitle{Spiegelungen}
	Spiegelungen im zweidimensionalen Raum:
	\begin{itemize}
		\item an x-Achse: \begin{gather*} \begin{pmatrix}
			1 & 0 \\
			0 & -1
			\end{pmatrix} \\ \end{gather*}
		\item an y-Achse:  \begin{gather*} \begin{pmatrix}
			-1 & 0 \\
			0 & 1
		\end{pmatrix} \\ \end{gather*}
		\item an der Diagonalen: \begin{gather*} \begin{pmatrix}
			0 & 1 \\
			1 & 0
			\end{pmatrix} \\ \end{gather*}
	\end{itemize}
\end{frame}

\begin{frame}
	\frametitle{Spiegelungen}
	Spiegelungen im dreidimensionalen Raum:
		\begin{gather*} x_{1}x_{2}-Ebene: \begin{pmatrix}
			1 & 0 & 0\\
			0 & 1 & 0\\
			0 & 0 & -1
			\end{pmatrix} \\
		x_{2}x_{3}-Ebene: \begin{pmatrix}
			-1 & 0 & 0\\
			0 & 1 & 0\\
			0 & 0 & 1
		\end{pmatrix} \\
		x_{1}x_{3}-Ebene: \begin{pmatrix}
			1 & 0 & 0\\
			0 & -1 & 0\\
			0 & 0 & 1
		\end{pmatrix} \\ \end{gather*}
\end{frame}

\begin{frame}
	\frametitle{Streckung / Stauchung}
	Prinzip für alle Räume identisch:
	\begin{itemize}
		\item in Richtung der x-Achse um Faktor s: \begin{gather*} \begin{pmatrix}
			s & 0 \\
			0 & 1
			\end{pmatrix} \\ \end{gather*}
		\item in Richtung der y-Achse um Faktor s: \begin{gather*} \begin{pmatrix}
			1 & 0 \\
			0 & s
		\end{pmatrix} \\ \end{gather*}
	\end{itemize}
\end{frame}

\begin{frame}
	\frametitle{Drehung}
	Drehung gegen den Uhrzeigersinn um Winkel $\alpha$:
	\begin{itemize}
		\item im zweidimensionalen: \begin{gather*} \begin{pmatrix}
			cos{\alpha} & -sin{\alpha} \\
			sin{\alpha} & cos{\alpha}
			\end{pmatrix} \end{gather*} \\
		\item im dreidimensionalen Raum wird immer um eine Achse gedreht:
	\end{itemize}
		\begin{gather*} um X_{1}-Achse \begin{pmatrix}
			1 & 0 & 0 \\
			0 & cos{\alpha} & -sin{\alpha} \\
			0 & sin{\alpha} & cos{\alpha}
		\end{pmatrix} \end{gather*} \\

\end{frame}

\begin{frame}
	\frametitle{Kombination von Transformationen}
	Es können auch mehrere Transforamtionen hintereinander angewendet werden:
	\begin{itemize}
		\item Es wird von rechts nach links gelesen
	\end{itemize}
	Beispiel:
	Erst Streckung der x-Achse, dann Spiegeln an Y-Achse:
	\begin{gather*}
	\begin{pmatrix}
		-1 & 0 \\
		0 & 1
	\end{pmatrix} \cdot
	\begin{pmatrix}
		2 & 0 \\
		0 & 1
	\end{pmatrix} \end{gather*}
\end{frame}

\begin{frame}
	\frametitle{Umkehrabbildung}
	Ist eine lineare Abbildung F umkehrbar (Streckung mit 2, zu Stauchung um $\frac{1}{2}$), dann gilt:
	\begin{itemize}
		\item Die Umkehrabbildung ist auch eine lineare Abbildung
		\item Die Umkehrungabbildung ist die Inverse von $F \rightarrow F^{-1}$
		\item Bei Inversen ist das Produkt von F und $F^{-1}$ die Einheitsmatrix
		\item Abbildungen bei denen sich die Dimension ändert sind nicht umkehrbar
	\end{itemize}
\end{frame}

\begin{frame}
	\frametitle{Affine Abbildungen}
	Eine Abbildung heißt affine Abbildung, wenn sie eine lineare Abbildung ist, und zusätzlich um einen Vektor verschoben wird.
	\begin{itemize}
		\item $F(A) = k*A + v$
		\item Dies ist keine lineare Abbildung
	\end{itemize}
\end{frame}

\begin{frame}
	\frametitle{Übungen}
	\begin{gather*}
	Vektor A: \begin{pmatrix}
		3 \\
		1 \\
		2
	\end{pmatrix}
	\end{gather*}
	\begin{itemize}
		\item A Spiegeln an der $x_{1}x_{2}$-Ebene
		\item A Strecken um den Faktor 5 in Richtung der $x_{2}$-Achse
		\item A drehen um 90° gegen den Uhrzeigersinn um die $x_{1}$-Achse
		\item A erst Strecken um den Faktor 5 in Richtung der $x_{3}$-Achse, dann Spiegeln an der $x_{2}x_{3}$-Ebene
	\end{itemize}
\end{frame}

\begin{frame}
	\frametitle{Lösungen}
	\begin{itemize}
		\item Spiegeln
		\begin{gather*}
		\begin{pmatrix}
			1 & 0 & 0 \\
			0 & 1 & 0 \\
			0 & 0 & $-1$
			\end{pmatrix} \cdot
			\begin{pmatrix}
				3 \\
				1 \\
				2
			\end{pmatrix} = \begin{pmatrix}
				3 \\
				1 \\
				-2
			\end{pmatrix} \end{gather*}
		\item Strecken
		\begin{gather*} \begin{pmatrix}
			1 & 0 & 0 \\
			0 & 5 & 0 \\
			0 & 0 & 1
			\end{pmatrix} \cdot
			\begin{pmatrix}
				3 \\
				1 \\
				2
			\end{pmatrix} = \begin{pmatrix}
				3 \\
				5 \\
				2
			\end{pmatrix} \end{gather*}
	\end{itemize}
\end{frame}

\begin{frame}
	\frametitle{Lösungen}
	\begin{itemize}
		\item Drehen
		\begin{gather*} \begin{pmatrix}
			1 & 0 & 0 \\
			0 & cos{90} & -sin{90} \\
			0 & sin{90} & cos{90}
			\end{pmatrix} \cdot
			\begin{pmatrix}
				3 \\
				1 \\
				2
				\end{pmatrix} = \begin{pmatrix}
					3 \\
					-2 \\
					1
				\end{pmatrix} \end{gather*}
		\item Strecken und Drehen
		\begin{gather*} \begin{pmatrix}
				-1 & 0 & 0 \\
				0 & 1 & 0 \\
				0 & 0 & 1
				\end{pmatrix} \cdot
				\begin{pmatrix}
					1 & 0 & 0 \\
					0 & 1 & 0 \\
					0 & 0 & 5
				\end{pmatrix} \cdot \begin{pmatrix}
					3 \\
					1 \\
					2
				\end{pmatrix} = \begin{pmatrix}
					-3 \\
					1 \\
					10
				\end{pmatrix} \end{gather*}
	\end{itemize}
\end{frame}