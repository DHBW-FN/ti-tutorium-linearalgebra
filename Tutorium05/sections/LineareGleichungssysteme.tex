\section{Lineare Gleichungssysteme}
\subsection{Allgemein}
\begin{frame}
    \frametitle{Allgemein}
    Ein lineares Gleichungssystem mit $n$ Unbekannten und $m$ Gleichungen ist gegeben durch
    \begin{equation*}
        \begin{cases}
            a_{11}x_1 + a_{12}x_2 + \dots + a_{1n}x_n = b_1 \\
            a_{21}x_1 + a_{22}x_2 + \dots + a_{2n}x_n = b_2 \\
            \vdots \\
            a_{m1}x_1 + a_{m2}x_2 + \dots + a_{mn}x_n = b_m
        \end{cases}
    \end{equation*}
    \begin{itemize}
        \item $a_{ij}$ sind die Koeffizienten der Gleichung
        \item $a_{ij},  b_i \in \mathbb{R}$
        \item Gilt $b_i = 0$ für alle $i$, so heißt das Gleichungssystem homogen
    \end{itemize}
\end{frame}

\begin{frame}
    \frametitle{Uebung}
    Welches Gleichungssystem ist linear?
    Falls ja, ist es auch homogen?
    \begin{enumerate}
        \item
        \begin{equation*}
            \begin{cases}
                x_1 + x_2 = 0 \\
                x_1 - x_2 = 0
            \end{cases}
        \end{equation*}
        \item
        \begin{equation*}
            \begin{cases}
                x_1 + x_2 = 0 \\
                x_1^2 + x_2^2 = 1
            \end{cases}
        \end{equation*}
        \item
        \begin{equation*}
            \begin{cases}
                x_1 + x_2 = 1 \\
                x_1 - x_2 = 1
            \end{cases}
        \end{equation*}
    \end{enumerate}
\end{frame}

\begin{frame}
    \frametitle{Uebung - Loesung}
    Welches Gleichungssystem ist linear?
    Falls ja, ist es auch homogen?
    \begin{enumerate}
        \item linear, homogen
        \begin{equation*}
            \begin{cases}
                x_1 + x_2 = 0 \\
                x_1 - x_2 = 0
            \end{cases}
        \end{equation*}
        \item nicht linear
        \begin{equation*}
            \begin{cases}
                x_1 + x_2 = 0 \\
                x_1^2 + x_2^2 = 1
            \end{cases}
        \end{equation*}
        \item linear, nicht homogen
        \begin{equation*}
            \begin{cases}
                x_1 + x_2 = 1 \\
                x_1 - x_2 = 1
            \end{cases}
        \end{equation*}
    \end{enumerate}
\end{frame}

\begin{frame}
    \frametitle{Zusammenhang LGS und Matrix}
    \begin{multline}
        \begin{cases}
            a_{11}x_1 + a_{12}x_2 + \dots + a_{1n}x_n = b_1 \\
            a_{21}x_1 + a_{22}x_2 + \dots + a_{2n}x_n = b_2 \\
            \vdots \\
            a_{m1}x_1 + a_{m2}x_2 + \dots + a_{mn}x_n = b_m
        \end{cases}
        \\ \equiv
        \begin{pmatrix}
            a_{11} & a_{12} & \dots  & a_{1n} \\
            a_{21} & a_{22} & \dots  & a_{2n} \\
            \vdots & \vdots & \ddots & \vdots \\
            a_{m1} & a_{m2} & \dots  & a_{mn}
        \end{pmatrix}
        \begin{pmatrix}
            x_1    \\
            x_2    \\
            \vdots \\
            x_n
        \end{pmatrix}
        =
        \begin{pmatrix}
            b_1    \\
            b_2    \\
            \vdots \\
            b_m
        \end{pmatrix}
    \end{multline}
    Die entstandene Matrix nennt man Koeffizientenmatrix
\end{frame}

\begin{frame}
    \frametitle{Loesung eines LGS}
    \begin{itemize}
        \item homogenes LGS hat genau eine(Nullvektor) oder unendlich viele Loesungen
        \item inhomogenes LGS kann keine, genau eine oder unendlich viele Loesungen haben
    \end{itemize}
\end{frame}

\subsection{Gauss-Jordan-Verfahren}
\begin{frame}
    \frametitle{Gauss-Jordan-Verfahren}
    \begin{itemize}
        \item Standardverfahren zur Loesung von LGS
        \item Erlaubte Operationen:
        \begin{enumerate}
            \item Vertauschen zweier Gleichungen
            \item Multiplikation einer Gleichung mit einer beliebigen nicht verschwindenden Zahl
            \item Addition einer Gleichung mit einer beliebigen Vielfachen einer anderen Gleichung
        \end{enumerate}
        \item Vorraussetzung: erweiterte Koeffizientenmatrix
        \item Ziel: Matrix in Zeilenstufenform bringen
        \begin{equation*}
            \begin{pmatrix}
                1      & a_{12} & \dots  & a_{1n} & b_1    \\
                0      & 1      & \dots  & b_{2n} & b_2    \\
                \vdots & \ddots & \vdots & \vdots & \vdots \\
                0      & \dots  & 0      & 1      & b_m
            \end{pmatrix}
        \end{equation*}
    \end{itemize}
\end{frame}

\begin{frame}
    \frametitle{Uebung}
    Loesen Sie das folgende Gleichungssystem mit dem Gauss-Jordan-Verfahren
    \begin{equation*}
        \begin{cases}
            x_1 + 2x_2 + 3x_3 = 1 \\
            2x_1 + 3x_2 + 4x_3 = 2 \\
            3x_1 + 4x_2 + 5x_3 = 3
        \end{cases}
    \end{equation*}
\end{frame}

\begin{frame}
    \frametitle{Uebung - Loesung}
    Loesen Sie das folgende Gleichungssystem mit dem Gauss-Jordan-Verfahren
    \begin{align*}
        &\begin{cases}
            x_1 + 2x_2 + 3x_3 = 1 \\
            2x_1 + 3x_2 + 4x_3 = 2 \\
            3x_1 + 4x_2 + 5x_3 = 3
        \end{cases} \\
        &\implies
        \begin{pmatrix}
            1 & 2 & 3 & \bigm| & 1 \\
            2 & 3 & 4 & \bigm| & 2 \\
            3 & 4 & 5 & \bigm| & 3
        \end{pmatrix}
        \implies
        \begin{pmatrix}
            1 & 2  & 3  & \bigm| & 1 \\
            0 & -1 & -2 & \bigm| & 0 \\
            0 & -2 & -3 & \bigm| & 0
        \end{pmatrix} \\
        &\implies
        \begin{pmatrix}
            1 & 2 & 3 & \bigm| & 1 \\
            0 & 1 & 2 & \bigm| & 0 \\
            0 & 2 & 3 & \bigm| & 0
        \end{pmatrix}
        \implies
        \begin{pmatrix}
            1 & 2 & 3 & \bigm| & 1 \\
            0 & 1 & 2 & \bigm| & 0 \\
            0 & 0 & 0 & \bigm| & 0
        \end{pmatrix}
    \end{align*}
\end{frame}