\section{Determinanten}
\begin{frame}
    \frametitle{Zweck / Wiederhohlung}
    Nicht jede Matrix hat eine Inverse (nicht regulär). \\
    Wenn die Determinante $\neq$ 0 dann ist die Matrix regulär. 
\end{frame}

\subsection{Regel von Laplace}

\begin{frame}
	\frametitle{Der Laplace'scher Entwicklungssatz}
	Die Determinante einer Matrix kann durch den Laplace'schen Entwicklungssatz \textbf{rekursiv} berechnet werden.
	\begin{gather*}
	\det(A) = \sum_{i=1}^n (-1)^{i+j} a_{ij} \det(A_{ij})
	\end{gather*}
\end{frame}

\begin{frame}
	 \frametitle{Laplace - Erklärung}
	Man sucht sich eine Spalte oder Zeile nach der entwickelt wird. 
	Dann wird für jeden Wert der Spalte oder Zeile die \textbf{Unterdeterminante} (Streichen der Spalte und Zeile der Zahl, bei (3x3)-Matrix $\rightarrow$ $\det(2x2)$ entwickelt.
	Das wird für alle Zahlen der gewählten Spalte oder Zeile gemacht.
	Die Zahlen erhalten nach dem Schachbrettmuster Vorzeichen und werden aufsummiert.
	\begin{gather*}
	\begin{pmatrix}
	+ & - & + & \dots \\
	- & + & - & \dots \\
	+ & - & + & \dots \\
	\vdots &  \vdots & \vdots & \vdots 
	\end{pmatrix}
	\end{gather*}
\end{frame}

\begin{frame}
	\frametitle{Beispiel}
	\begin{gather*}
	\det(A) = \det\begin{vmatrix}
	1 & 2 & 3 \\
	4 & 5 & 6 \\
	7 & 8 & 9
	\end{vmatrix} \\
	= 1 \cdot \det\begin{vmatrix}
	5 & 6 \\
	8 & 9
	\end{vmatrix} - 2 \cdot \det\begin{vmatrix}
	4 & 6 \\
	7 & 9
	\end{vmatrix} + 3 \cdot \det\begin{vmatrix}
	4 & 5 \\
	7 & 8
	\end{vmatrix}\\
	= 1 \cdot (5 \cdot 9 - 6 \cdot 8) - 2 \cdot (4 \cdot 9 - 6 \cdot 7) + 3 \cdot (4 \cdot 8 - 5 \cdot 7) = 0
	\end{gather*}
\end{frame}


\begin{frame}
	\frametitle{Übungen}
	Berechne die Derminante der Folgenden Matrizen:
	\begin{enumerate}
	\item A = $\begin{pmatrix}
		2 & 1 & 0 \\
		5 & 4 & -3 \\
		2 & -1 & 6
		\end{pmatrix}$
	\item B = $\begin{pmatrix}
		3 & 5 & 1 & -4 \\
		-4 & 2 & 0 & 0 \\
		3 & 8 & 1 & 1 \\
		1 & 2 & 0 & 2
		\end{pmatrix}$
		\end{enumerate}
\end{frame}

\begin{frame}
	\frametitle{Übungen - Lösung A}
	\begin{gather*}
	 \det(A) = \begin{vmatrix}
		2 & 1 & 0 \\
		5 & 4 & -3 \\
		2 & -1 & 6
		\end{vmatrix} 
		= 2 \cdot \begin{vmatrix}
		4 & -3 \\
		-1 & 6
		\end{vmatrix} - 1 \cdot \begin{vmatrix}
		5 & -3 \\
		2 & 6
		\end{vmatrix}\\
		= 2 \cdot (4 \cdot 6 - (-3) \cdot (-1)) - 1 \cdot (5 \cdot 6 - (-3) \cdot 2) = 6
	\end{gather*}
\end{frame}

\begin{frame}
	\frametitle{Übungen - Lösung B}
	\begin{gather*}
	 \det(B) = \begin{vmatrix}
		3 & 5 & 1 & -4 \\
		-4 & 2 & 0 & 0 \\
		3 & 8 & 1 & 1 \\
		1 & 2 & 0 & 2
		\end{vmatrix} 
		= 1 \cdot \begin{vmatrix}
		-4 & 2 & 0 \\
		3 & 8 & 1 \\
		1 & 2 & 2
		\end{vmatrix} + 1 \cdot \begin{vmatrix}
		3 & 5 & -4 \\
		-4 & 2 & 0 \\
		1 & 2 & 2
		\end{vmatrix} \\
		= -4 \cdot \begin{vmatrix} 8 & 1 \\ 2 & 2 \end{vmatrix}
		-2 \cdot \begin{vmatrix} 3 & 1 \\  1 & 2 \end{vmatrix}
		+4 \cdot \begin{vmatrix} 5 & -4 \\ 2 & 2 \end{vmatrix}
		+2 \cdot \begin{vmatrix} 3 & -4 \\ 1 & 2 \end{vmatrix}\\
		= -4 \cdot (8 \cdot 2 - 2 \cdot 1)
		-2 \cdot (3 \cdot 2 - 1 \cdot 1) \\
		+4 \cdot (5 \cdot 2 - 2 \cdot (-4))
		+ 2 \cdot (3 \cdot 2 - 1 \cdot (-4)) \\
		= 26
	\end{gather*}
\end{frame}

\subsection{Sätze}
\begin{frame}
	\frametitle{Sätze}
	\textbf{Satz 1:} \\
	Für quadratische Matrizen A und B gilt für das Produkt $\det(AB)$ = $\det(A) \cdot \det(B)$. Ist A invertierbar, so folgt daraus $\det(A-1)$ = $\det(A)-1$. \\
	$ $ \\
	\textbf{Satz 2:} \\
	Eine quadratische Matrix A ist genau dann invertierbar, wenn $\det(A)$ $\neq$ 0 gilt.
2. Die Spaltenvektoren einer quadratischen Matrix A sind genau dann linear unabhängig, wenn
$\det(A)$ $\neq$ 0 gilt. \\
\end{frame}


\subsection{Zeilenstufenform}
\begin{frame}
	\frametitle{Zeilenstufenform}
	Die Determinante einer Dreiecksmatrix ist das Produkt ihrer Diagonalelemente.

	Man kann eine Matrix zur Dreiecksmatrix umformen. 
	Folgende Eigentschaften müssen berücksichtigt werden.
	\begin{enumerate}
	\item Vertauschen zweier Spalten/Zeilen $\rightarrow$ ändert Vorzeichen
	\item Multiplizieren einer Spalte/Zeile mit einer Zahl k $\rightarrow$ multipliziert Determinante mit k
	\item Addition des Vielfachen einer Zeile/Spalte mit einer anderen Zeile/Spalte ändert Determinante \textbf{nicht}.
	\end{enumerate}
	Das heißt, wenn ihr Multipliziert oder Vertauscht müsst ihr auch das Aufschreiben und am Ende zurückrechnen.
\end{frame}


\subsection{Regel von Sarrus}

\begin{frame}
	\frametitle{Regel von Sarrus}
	Der Satz von Sarrus ist eine einfache Methode um Determinanten für (3,3)-Matrizen zu berechnen.

	\begin{gather*}
	\begin{vmatrix}
	a & b & c \\
	d & e & f \\
	g & h & i
	\end{vmatrix} = aei + bfg + cdh - ceg - afh - bdi 
	\end{gather*}
\end{frame}



\begin{frame}
	\frametitle{Beispiel}
	\begin{gather*}
		det(A) = \begin{vmatrix}
			1 & 2 & 3 \\
			4 & 5 & 6 \\
			7 & 8 & 9
		\end{vmatrix} \\
		= 1 \cdot 5 \cdot 9 + 2 \cdot 6 \cdot 7 + 3 \cdot 4 \cdot 8 - 3 \cdot 5 \cdot 7 - 2 \cdot 4 \cdot 9 - 1 \cdot 6 \cdot 8 = 0
	\end{gather*}
\end{frame}

\begin{frame}
	\frametitle{Übungen}
	\begin{enumerate}
	\item A = $\begin{pmatrix}
	0 & 1 & 2 \\
	3 & 2 & 1 \\
	1 & 1 & 0
	\end{pmatrix}$
	\end{enumerate}
\end{frame}

\begin{frame}
	\frametitle{Übungen - Lösung}
	\begin{gather*}
	 \det(A) =  \\
	 0 \cdot 2 \cdot 0 
	 + 1 \cdot 1 \cdot 1 
	 + 2 \cdot 3 \cdot 1
	 - 2 \cdot 2 \cdot 1
	 - 0 \cdot 1 \cdot 1
	 - 1 \cdot 3 \cdot 0 \\
	 = 3
	\end{gather*}
\end{frame}


