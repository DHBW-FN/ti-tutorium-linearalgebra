\section{Skalarprodukt und Orthogonalität}

\subsection{Skalarprodukt}
\begin{frame}
    \frametitle{Skalarprodukt}
    Das Skalarprodukt ordnet zwei Vektoren ein Skalar zu.
    \begin{itemize}
	    \item Multiplikation zweier Vektoren $\rightarrow$ Skalar (Reele Zahl)
	    \item Kann nur gebildet werden, wenn die Vektoren die gleiche Dimension haben
	    \item Wenn das Skalar = 0 ist, sind die Vektoren orthogonal (Rechter Winkel)
    \end{itemize}
	Beispiel: $\vec{a} = (1,2,3)$ und $\vec{b} = (4,5,6)$
	\begin{align*}
		\vec{a} \bullet \vec{b} &= (1,2,3) \bullet (4,5,6) \\
		&= 1 \cdot 4 + 2 \cdot 5 + 3 \cdot 6 \\
		&= 32
	\end{align*}
\end{frame}


\begin{frame}
    \frametitle{Skalarprodukt Übungen}
	\begin{enumerate}
		\item $\vec{a} \bullet \vec{b} = (0, -3\pi) \bullet (\sqrt{2}, 0)$ 
		\item $\vec{a} \bullet \vec{b} = (-21, 5) \bullet (\sqrt{5}, -3, 1)$
	\end{enumerate}
\end{frame}

\begin{frame}
    \frametitle{Skalarprodukt Lösungen}
	\begin{enumerate}
		\item $\vec{a} \bullet \vec{b} = (0, -3\pi) \bullet (\sqrt{2}, 0) = 0$ 
		\item $\vec{a} \bullet \vec{b} = (-21, 5) \bullet (\sqrt{5}, -3, 1)$ \\andere Dimension nicht möglich
	\end{enumerate}
\end{frame}

\subsection{Betrag eines Vektors}
\begin{frame}
    \frametitle{Betrag eines Vektors}
    Der Betrag eines Vektors ist die Länge des Vektors.
    \begin{itemize}
	    \item Die länge eines Vektors ist immer positiv
	    \item Der Betrag eines Vektors wird mit $\lVert \vec{a} \rVert$ dargestellt
	\item $\lVert \vec{a} \rVert = \sqrt{<\vec{a}, \vec{a}>}$
    \end{itemize}
        Beispiel: $\vec{a} = (1,2,3)$
	\begin{align*}
		\lVert \vec{a} \rVert &= \sqrt{1^2 + 2^2 + 3^2} \\
		&= \sqrt{14}
	\end{align*}
\end{frame}

\begin{frame}
    \frametitle{Vektor Betrag Übungen}
	Berechnen Sie den Betrag der Vektoren:
	\begin{enumerate}
		\item $\lVert \begin{pmatrix} -1 \\ 5 \\ -0.5 \end{pmatrix} \rVert$
	\end{enumerate}
\end{frame}

\begin{frame}
    \frametitle{Vektor Betrag Übungen}
	Berechnen Sie den Betrag der Vektoren:
	\begin{enumerate}
		\item $\lVert \begin{pmatrix} -1 \\ 5 \\ -0.5 \end{pmatrix} \rVert = $ \\
		$ $\\
		$ \sqrt{(-1)^2 + 5^2 + (-0.5)^2} = \sqrt{26.25} = 5.124$
	\end{enumerate}
\end{frame}

\subsection{Winkel zwischen Vektoren}
\begin{frame}
    \frametitle{Winkel zwischen Vektoren}
    Der Winkel zwischen zwei Vektoren ist definiert über:
	\begin{align*}
		    \cos \varphi &= \frac{\vec{a} \bullet \vec{b}}{\lVert \vec{a} \rVert \cdot \lVert \vec{b} \rVert} \\
		    \varphi &= \arccos \left( \frac{\vec{a} \bullet \vec{b}}{\lVert \vec{a} \rVert \cdot \lVert \vec{b} \rVert} \right)
	\end{align*}
	\begin{itemize}
		\item Wenn der Winkel 0° ist, sind die Vektoren parallel
		\item Wenn der Winkel 90° ist sind die Vektoren orthogonal
		\item Achtung: Je nach Taschenrechner Modus kommen unterschiedliche Werte raus
		\item Um den Winkel zu bekommen stellt den Rechner auf (D)egree
	\end{itemize}
\end{frame}


