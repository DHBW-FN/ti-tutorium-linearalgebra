\section{Skalarprodukt und Orthogonalität}
\subsection{Skalarprodukt}
\begin{frame}
    \frametitle{Skalarprodukt}
    Das Skalarprodukt ordnet zwei Vektoren ein Skalar zu.
    \begin{itemize}
	    \item Multiplikation zweier Vektoren $\rightarrow$ Skalar (meist Reelle Zahl)
	    \item Kann nur gebildet werden, wenn die Vektoren die gleiche Dimension haben
	    \item Wenn das Skalar = 0 ist, sind die Vektoren orthogonal (Rechter Winkel)
    \end{itemize}
	Beispiel: $\vec{a} = (1,2,3)$ und $\vec{b} = (4,5,6)$
	\begin{align*}
		\vec{a} \bullet \vec{b} &= (1,2,3) \bullet (4,5,6) \\
		&= 1 \cdot 4 + 2 \cdot 5 + 3 \cdot 6 \\
		&= 32
	\end{align*}
\end{frame}


\begin{frame}
    \frametitle{Skalarprodukt Übungen}
	\begin{enumerate}
		\item $\vec{a} \bullet \vec{b} = (0, -3\pi) \bullet (\sqrt{2}, 0)$ 
		\item $\vec{a} \bullet \vec{b} = (-21, 5) \bullet (\sqrt{5}, -3, 1)$
	\end{enumerate}
\end{frame}

\begin{frame}
    \frametitle{Skalarprodukt Lösungen}
	\begin{enumerate}
		\item $\vec{a} \bullet \vec{b} = (0, -3\pi) \bullet (\sqrt{2}, 0) = 0$ 
		\item $\vec{a} \bullet \vec{b} = (-21, 5) \bullet (\sqrt{5}, -3, 1)$ \\andere Dimension nicht möglich
	\end{enumerate}
\end{frame}

\subsection{Betrag eines Vektors}
\begin{frame}
    \frametitle{Betrag eines Vektors}
    Der Betrag eines Vektors ist die Länge des Vektors.
    \begin{itemize}
	    \item Die länge eines Vektors ist immer positiv
	    \item Der Betrag eines Vektors wird mit $\lVert \vec{a} \rVert$ dargestellt
	\item $\lVert \vec{a} \rVert = \sqrt{<\vec{a}, \vec{a}>}$
    \end{itemize}
        Beispiel: $\vec{a} = (1,2,3)$
	\begin{align*}
		\lVert \vec{a} \rVert &= \sqrt{1^2 + 2^2 + 3^2} \\
		&= \sqrt{14}
	\end{align*}
\end{frame}

\begin{frame}
    \frametitle{Vektor Betrag Übungen}
	Berechnen Sie den Betrag der Vektoren:
	\begin{enumerate}
		\item $\lVert \begin{pmatrix} -1 \\ 5 \\ -0.5 \end{pmatrix} \rVert$
	\end{enumerate}
\end{frame}

\begin{frame}
    \frametitle{Vektor Betrag Übungen}
	Berechnen Sie den Betrag der Vektoren:
	\begin{enumerate}
		\item $\lVert \begin{pmatrix} -1 \\ 5 \\ -0.5 \end{pmatrix} \rVert = $ \\
		$ $\\
		$ \sqrt{(-1)^2 + 5^2 + (-0.5)^2} = \sqrt{26.25} = 5.124$
	\end{enumerate}
\end{frame}

\subsection{Winkel zwischen Vektoren}
\begin{frame}
    \frametitle{Winkel zwischen Vektoren}
    Der Winkel zwischen zwei Vektoren ist definiert über:
	\begin{align*}
		    \cos \varphi &= \frac{\vec{a} \bullet \vec{b}}{\lVert \vec{a} \rVert \cdot \lVert \vec{b} \rVert} \\
		    \varphi &= \arccos \left( \frac{\vec{a} \bullet \vec{b}}{\lVert \vec{a} \rVert \cdot \lVert \vec{b} \rVert} \right)
	\end{align*}
	\begin{itemize}
		\item Wenn der Winkel 0° ist, sind die Vektoren parallel
		\item Wenn der Winkel 90° oder $\frac{\pi}{2}$ ist sind die Vektoren orthogonal
		\item \textbf{Achtung:} Je nach Taschenrechner Modus kommen unterschiedliche Werte raus
		\item Um den Winkel zu bekommen stellt den Rechner auf (D)egree
	\end{itemize}
\end{frame}

\begin{frame}
	\frametitle{Winkel zwischen Vektoren Beispiel}
	Beispiel: $\vec{a} = (1,2)$ $\vec{b} = (-0.5, 3)$ \\
	$ $ \\
	Skalarprodukt von $\vec{a}$ und $\vec{b}$ \\
	$\vec{a} \bullet \vec{b} = (1,2) \bullet (-0.5, 3) = 5.5$ \\
	$ $ \\
	Betrag von $\vec{a}$ und $\vec{b}$ \\
	$\lVert \vec{a} \rVert = \sqrt{1^2 + 2^2} = \sqrt{5}$ \\
	$\lVert \vec{b} \rVert = \sqrt{(-0.5)^2 + 3^2} = \sqrt{9.25}$ \\
	$ $ \\
	Winkel zwischen $\vec{a}$ und $\vec{b}$ \\ 
	$\varphi = \arccos(\frac{5.5}{\sqrt{5} \cdot \sqrt{9.25}}) = \arccos(0.81) = 36$°
\end{frame}

\begin{frame}
	\frametitle{Winkel zwischen Vektoren Übungen}
	Berechnen Sie den Winkel zwischen den Vektoren
	\begin{enumerate}
		\item $\vec{a} = (2\pi, 7)$ $\vec{b} = (-3.5, \pi)$
		\item $\vec{a} = (-2, 7)$ $\vec{b} = (5, 3)$
	\end{enumerate}
\end{frame}

\begin{frame}
	\frametitle{Winkel zwischen Vektoren Lösungen}
	1. $\vec{a} = (2\pi, 7)$ $\vec{b} = (-3.5, \pi)$ \\
	Skalarprodukt von $\vec{a}$ und $\vec{b}$ \\
	$\vec{a} \bullet \vec{b} = 2\pi \cdot (-3.5) + 7 \cdot \pi = 7\pi - 7\pi = 0$ \\
	$\rightarrow$ Somit sind die Vektoren orthogonal und der Winkel ist 90° \\
	$ $ \\
	2. $\vec{a} = (-2, 7)$ $\vec{b} = (5, 3)$
	Skalarprodukt von $\vec{a}$ und $\vec{b}$ \\
	$\vec{a} \bullet \vec{b} = -2 \cdot 5 + 7 \cdot 3 = 11$ \\
	Betrag von $\vec{a}$ und $\vec{b}$ \\
	$\lVert \vec{a} \rVert = \sqrt{(-2)^2 + 7^2} = \sqrt{53}$ \\
	$\lVert \vec{b} \rVert = \sqrt{5^2 + 3^2} = \sqrt{34}$ \\
	Winkel zwischen $\vec{a}$ und $\vec{b}$ \\
	$\varphi = \arccos(\frac{11}{\sqrt{53} \cdot \sqrt{34}}) = 75$°
\end{frame}
		
\subsection{Orthogonale Projektion}
\begin{frame}
	\frametitle{Orthogonale Projektion}
	Gegeben sind ein Vektor $\vec{a}$ und ein Einheitsvektor $\vec{e}$\\
	$\vec{a}$ kann dann eindeutig in zwei zueinander orthogonale Komponenten zerlegt werden:\\
	$\vec{a} = \vec{a}_{\parallel} + \vec{a}_{\perp}$ \\
	$ $\\
	$\vec{a}_{\parallel} = < \vec{a}, \vec{e}> \cdot \vec{e}$ \\
	\begin{itemize}
		\item orthogonale Projektion von $\vec{a}$ in Richtung $\vec{e}$ 
		\item parrallel zu $\vec{e}$
	\end{itemize}
	$\vec{a}_{\perp} = \vec{a} - <\vec{a}, \vec{e}> \cdot \vec{e} = \vec{a} - \vec{a}_{\parallel}$
	\begin{itemize}
		\item orthogonales Komplement von $\vec{a}$ in Richtung $\vec{e}$
		\item ist orthogonal zu $\vec{e}$
	\end{itemize}
\end{frame}

\begin{frame}
	\frametitle{Orthogonale Projektion Beispiel}
	Vektor $\vec{a} = (2, -2, 3)$, Einheitsvektor $\vec{e} = \frac{1}{\sqrt{3}}(1, 1, 1)$\\

    \begin{enumerate}
        \item Bilden von $\vec{a}_{\parallel}$\\
        $\vec{a}_{\parallel} = \frac{1}{\sqrt{3}} \cdot <(2, -2, 3), (1, 1, 1)> \cdot \frac{1}{\sqrt{3}} \cdot (1, 1, 1)$
        $= \frac{1}{3} \cdot 3 \cdot (1, 1, 1,) = (1, 1, 1)$
        \item Bilden von $\vec{a}_{\perp}$\\
        $\vec{a}_{\perp} = \vec{a} - \vec{a}_{\parallel} = (2, -2, 3) - (1, 1, 1) = (1, -3 , 2)$
        \item Überprüfung\\
        $\vec{a} = \vec{a}_{\parallel} + \vec{a}_{\perp} = (1, 1, 1) + (1, -3, 2) = (2, -2, 3)$
    \end{enumerate}
\end{frame}
	

\begin{frame}
	\frametitle{Orthogonale Projektion Übung}
	Übung Orthogonale Projektion \\
	$\vec{a} = (2, 5)$ $\vec{b} = (-2, 3)$\\
	Orthogonale Zerlegung von $\vec{a}$ in Richtung von $\vec{b}$
\end{frame}


\begin{frame}
	\frametitle{Orthogonale Projektion Lösung}
	Lösung Orthogonale Projektion\\
    \begin{enumerate}
        \item Bilden des Einheitsvektors \\
        $\vec{e} = \frac{\vec{b}}{\lvert \vec{b} \rvert} = \frac{1}{\sqrt{13}} \cdot (-2, 3)$\\
        \item Bilden von $\vec{a}_{\parallel}$
        $\vec{a}_{\parallel} = \frac{1}{\sqrt{13}} \cdot <(-2, 3), (2, 5)> \cdot \frac{1}{\sqrt{13}} \cdot (-2, 3)$ \\
        $= \frac{1}{13} \cdot 11 \cdot (-2, 3) = \frac{11}{13}(-2, 3)$\\
        \textit{Faktor von $\vec{e}$  aus Skalarprodukt ausgeklammert}\\
        \item Bilden von $\vec{a}_{\perp}$ \\
        $\vec{a}_{\perp} = \vec{a} - \vec{a}_{\parallel} = (2, 5) - \frac{11}{13} \cdot (-2, 3)$\\
        $= (\frac{48}{13}, \frac{32}{13}) = \frac{8}{13}(6, 4)$\\
        \item Überprüfung \\
        $\vec{a} = \vec{a}_{\parallel} + \vec{a}_{\perp}$\\
        $\frac{11}{13}(-2, 3) + \frac{8}{13}(6, 4) = (2, 5)$ \\
    \end{enumerate}
\end{frame}
