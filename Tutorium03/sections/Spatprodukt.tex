\section{Spatprodukt}
\begin{frame}
    \frametitle{Spatprodukt}
    Das Spatprodukt bildet einen Körper mit der Grundfläche eines Prallelogramms (Parallelepiped).
    \begin{itemize}
        \item Aufgespannt von 3 Vektoren
        \item Zwei davon im Kreuzprodukt (Prallelogramm)
        \item Höhe über dem Prallelogramm durch Skalarprodukt
        \item $V = \vec{a} \times \vec{b} \cdot \vec{c}$
        \item Die reihenfolge der Vektoren spielt keine Rolle
    \end{itemize}
\end{frame}

\begin{frame}
    \frametitle{Spatprodukt  - Matrix}
    Alternativ kann das Spatprodukt über die Determinante der Matrix berechnet werden
    \begin{itemize}
        \item Alle Vektoren in eine $3 \times 3$ Matrix schreiben
        \item Determinante über den Satz von Sarrus berechnen
        \item Determinante ist das Spatprodukt und damit das Volumen des Parallelepiped
    \end{itemize}
\end{frame}