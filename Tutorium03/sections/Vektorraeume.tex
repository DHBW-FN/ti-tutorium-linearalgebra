\section{Vektorraeume}

\subsection{Vektoren}
\begin{frame}
    \frametitle{Vektoren Grundlagen}
    Ein Vektor ist ein Tupel aus $n$ reellen Zahlen, also ein $n$-dimensionaler Vektor.
    \begin{itemize}
        \item $n$-dimensionaler Vektor: $v = (v_1, v_2, \dots, v_n)$
        \item $n$-dimensionaler Vektor: $v = \begin{pmatrix}
                                                 v_1 \\ v_2 \\ \vdots \\ v_n
        \end{pmatrix}$
        \item $n$-dimensionaler Vektor: $v = \begin{pmatrix}
                                                 v_1 & v_2 & \dots & v_n
        \end{pmatrix}$
    \end{itemize}
\end{frame}

\begin{frame}
    \frametitle{Mit Vektoren rechnen}
    Seien v= $(v_1, v_2, \dots, v_n)$ und w= $(w_1, w_2, \dots, w_n)$ zwei $n$-dimensionale Vektoren.
    \begin{itemize}
        \item Addition: $v + w = (v_1 + w_1, v_2 + w_2, \dots, v_n + w_n)$
        \item Subtraktion: $v - w = (v_1 - w_1, v_2 - w_2, \dots, v_n - w_n)$
        \item Skalarmultiplikation: $c \cdot v = (c \cdot v_1, c \cdot v_2, \dots, c \cdot v_n)$
        \item Länge: $|v| = \sqrt{v_1^2 + v_2^2 + \dots + v_n^2}$
        \item Normalisierung: $\hat{v} = \frac{v}{|v|}$
        \item Skalarprodukt: $v \cdot w = v_1 \cdot w_1 + v_2 \cdot w_2 + \dots + v_n \cdot w_n$
    \end{itemize}
\end{frame}

\begin{frame}
    \frametitle{Vektoren Uebung}
    Seien $v = (-2, 3, -6)$ und $w = (4, 5, 6)$ zwei $n$-dimensionale Vektoren.
    Berechnen Sie:
    \begin{enumerate}
        \item $v + w$.
        \item $v - w$.
        \item $2 \cdot v$.
        \item $|v|$.
        \item $\hat{v}$.
        \item $v \cdot w$.
    \end{enumerate}
\end{frame}

\begin{frame}
    \frametitle{Vektoren Uebung Loesung}
    Seien $v = (-2, 3, -6)$ und $w = (4, 5, 6)$ zwei $n$-dimensionale Vektoren.
    \begin{enumerate}
        \item $v + w = (-2 + 4, 3 + 5, -6 + 6) = (2, 8, 0)$.
        \item $v - w = (-2 - 4, 3 - 5, -6 - 6) = (-6, -2, -12)$.
        \item $2 \cdot v = (2 \cdot -2, 2 \cdot 3, 2 \cdot -6) = (-4, 6, -12)$.
        \item $|v| = \sqrt{(-2)^2 + 3^2 + (-6)^2} = \sqrt{4 + 9 + 36} = \sqrt{49} = 7$.
        \item $\hat{v} = \frac{v}{|v|} = \frac{(-2, 3, -6)}{7} = \frac{(-2, 3, -6)}{7} = (-\frac{2}{7}, \frac{3}{7}, -\frac{6}{7})$.
        \item $v \cdot w = ((-2) \cdot 4) + (3 \cdot 5) + ((-6) \cdot 6) = (-8) + 15 + (-36) = -29$.
    \end{enumerate}
\end{frame}

\begin{frame}
    \frametitle{Besondere Vektoren}
    \begin{itemize}
        \item Nullvektor: $v = (0, 0, \dots, 0)$
        \item Einheitsvektor: $|v| = 1$
        \item Vektor der Standardbasis: $e_i = (0, 0, \dots, 1, \dots, 0)$
    \end{itemize}
\end{frame}

\subsection{Exkurs}
\begin{frame}
    \frametitle{Exkurs: Mathematische Definition Vektorräume}
    \begin{itemize}
        \item Gruppe
        \begin{itemize}
            \item $(G,*)$, Menge $G$, Verknüpfung $*$
            \item Assoziativität
            \item Existenz eines neutrales Elements
            \item Existenz eines inversen Elements
        \end{itemize}
        \item Ring
        \begin{itemize}
            \item $(R,+,*)$ mit $R$ Menge, $+$ Verknüpfung, $*$ Verknüpfung
            \item $(R,+)$ ist abelsche Gruppe(Gruppe mit Kommutativgesetz)
            \item $(R,*)$ ist Halbgruppe(nur Assoziativität)
            \item Distributiv 1: $a * (b + c) = a * b + a * c$
            \item Distributiv 2: $(a + b) * c = a * c + b * c$
        \end{itemize}
    \end{itemize}
\end{frame}

\begin{frame}
    \frametitle{Exkurs: Mathematische Definition Vektorräume}
    \begin{itemize}
        \item Körper
        \begin{itemize}
            \item $(K,+,*)$ mit $K$ Menge, $+$ Verknüpfung, $*$ Verknüpfung
            \begin{itemize}
                \item $(K,+)$ ist abelsche Gruppe
                \item $(K\setminus\{0\},*)$ ist abelsche Gruppe
                \item Distributiv 1: $a * (b + c) = a * b + a * c$
                \item Distributiv 2: $(a + b) * c = a * c + b * c$
            \end{itemize}
            \item alternativ: kommutativer unitärer Ring, der nicht Nullring ist
        \end{itemize}
    \end{itemize}
\end{frame}

\begin{frame}
    \frametitle{Exkurs: Ursprung der Vektorraumaxiome}
    \begin{itemize}
        \item K-Vektorraum
        \begin{itemize}
            \item $(V,+,*)$
            \item $(V,+)$ ist abelsche Gruppe
            \item $\cdot: K \times V \rightarrow V ((\lambda, v) \mapsto \lambda \cdot v)$
            \begin{itemize}
                \item $\forall v,w \in V$ und $\forall \lambda, \mu \in K$ gilt:
                \item $(\lambda + \mu) \cdot v = \lambda \cdot v + \mu \cdot v$
                \item $\lambda \cdot (v + w) = \lambda \cdot v + \lambda \cdot w$
                \item $(\lambda \cdot \mu) \cdot v = \lambda \cdot (\mu \cdot v)$
                \item $\lambda \cdot 1 = \lambda$
            \end{itemize}
        \end{itemize}
    \end{itemize}
\end{frame}

\subsection{Vektorräume}
\begin{frame}
    \frametitle{Vektorräume}
    Ein Vektorraum ist eine Menge von Vektoren, die miteinander addiert oder mit Skalaren multipliziert werden können.
\end{frame}

\begin{frame}
    \frametitle{Allgemeine Rechenregeln}
    Vektoraddition
    \begin{itemize}
        \item Kommutativgesetz: $v + w = w + v$
        \item Assoziativgesetz: $(v + w) + u = v + (w + u)$
        \item Existenz des neutralen Elements: $v + 0 = v$
        \item Existenz des inversen Elements: $v + (-v) = 0$
    \end{itemize}

    Skalarmultiplikation
    \begin{itemize}
        \item Distributivgesetz 1: $c \cdot (v + w) = c \cdot v + c \cdot w$
        \item Distributivgesetz 2: $(c + d) \cdot v = c \cdot v + d \cdot v$
    \end{itemize}
\end{frame}
