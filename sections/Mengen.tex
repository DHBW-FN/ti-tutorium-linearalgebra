\section{Mengen}
\begin{frame}
    \frametitle{Wiederholung}
    \begin{itemize}
        \item Eine \textbf{Menge} ist eine Zusammenfassung wohlunterschiedener Objekte
        \item Einzelne Objekte heißen \textbf{Elemente}.
        \item A ist \textbf{Teilmenge} von B, wenn jedes Element von A auch Element von B ist.
    \end{itemize}
\end{frame}

\begin{frame}
    \frametitle{Mengenschreibweisen}
    \begin{itemize}
        \item $A = \{a_1, a_2, \dots, a_n\} $ bedeutet die Menge aller Elemente $a_1, a_2, \dots, a_n$
        \item $A = \{a \in \mathbb{N} \mid a \bmod 2 = 0\}$ bedeutet die Menge aller natürlichen Zahlen, die durch 2 teilbar sind
        \item $ A \subseteq B $ bedeutet, dass $ A $ Teilmenge von $ B $ ist
        \item $ A \subset B $ bedeutet, dass $ A $ Teilmenge von $ B $ ist und $ A \neq B $
        \item $a \in A$ bedeutet, dass $a$ Element von $A$ ist
        \item $a \notin A$ bedeutet, dass $a$ kein Element von $A$ ist
    \end{itemize}
\end{frame}

\begin{frame}
    \frametitle{Mengenoperationen}
    \begin{itemize}
        \item $ A \subseteq B $ bedeutet, dass $ A $ Teilmenge von $ B $ ist
        \item $ A \subset B $ bedeutet, dass $ A $ Teilmenge von $ B $ ist und $ A \neq B $
        \item $a \in A$ bedeutet, dass $a$ Element von $A$ ist
        \item $a \notin A$ bedeutet, dass $a$ kein Element von $A$ ist
        \item $A \cap B$ bedeutet die Schnittmenge von $A$ und $B$
        \item $A \cup B$ bedeutet die Vereinigungsmenge von $A$ und $B$
        \item $A \setminus B$ bedeutet die Differenzmenge von $A$ und $B$
    \end{itemize}
\end{frame}

\begin{frame}
    \frametitle{Uebung}
    Es seien $A = \{1,2,3,4,5\}$ und $B = \{3,4,5,6,7\}$ zwei Mengen.
    Berechnen Sie die folgenden Mengen:
    \begin{enumerate}
        \item $A \cap B$
        \item $A \cup B$
        \item $A \setminus B$
        \item $B \setminus A$
    \end{enumerate}
\end{frame}

\begin{frame}
    \frametitle{Loesung}
    Es seien $A = \{1,2,3,4,5\}$ und $B = \{3,4,5,6,7\}$ zwei Mengen.
    Berechnen Sie die folgenden Mengen:
    \begin{enumerate}
        \item $A \cap B = \{3,4,5\}$
        \item $A \cup B = \{1,2,3,4,5,6,7\}$
        \item $A \setminus B = \{1,2\}$
        \item $B \setminus A = \{6,7\}$
    \end{enumerate}
\end{frame}

\begin{frame}
    \frametitle{Bekannte Zahlenmengen}
    \begin{itemize}
        \item $\mathbb{N} = \{0,1,2,3,4, \dots\}$ ist die Menge aller natürlichen Zahlen
        \item $\mathbb{Z} = \{\dots, -2, -1, 0, 1, 2, \dots\}$ ist die Menge aller ganzen Zahlen
        \item $\mathbb{Q} = \{\frac{a}{b} \mid a \in \mathbb{Z}, b \in \mathbb{N} \wedge b \neq 0\}$ ist die Menge aller rationalen Zahlen
        \item $\mathbb{I} = \mathbb{R} \cup \mathbb{Q}$ ist die Menge aller irrationalen Zahlen
        \item $\mathbb{R} = \mathbb{Q} \cup \mathbb{I}$ ist die Menge aller reellen Zahlen
        \item $\mathbb{C} = \{a + bi \mid a, b \in \mathbb{R}\}$ ist die Menge aller komplexen Zahlen $i = \sqrt{-1}$
    \end{itemize}
\end{frame}
