\section{Wiederholung}
\begin{frame}
    \frametitle{Bruchgesetze}
    \begin{enumerate}
        \setlength\itemsep{1em}
        \item $\frac{a}{b}=\frac{ac}{bc}$
        \item $\frac{a}{b}=\frac{a \div c}{b \div c}$
        \item $\frac{a}{b}*c=\frac{ac}{b}$
        \item $\frac{a}{b}*c^{-1}=\frac{a}{bc}$
        \item $\frac{a}{b} \div c=\frac{a}{bc}$
    \end{enumerate}
\end{frame}

\begin{frame}
    \frametitle{Bruchgesetze Rechenarten}
    \begin{enumerate}
        \setlength\itemsep{1em}
        \item $\frac{a}{b} + \frac{c}{b}=\frac{a+c}{b}$
        \item $\frac{a}{b} - \frac{c}{b}=\frac{a-c}{b}$
        \item $\frac{a}{b} + \frac{c}{d}=\frac{ad+bc}{bd}$
        \item $\frac{a}{b} = \frac{1}{\frac{b}{a}}$
        \item $\frac{a}{b} \div \frac{c}{d} = \frac{\frac{a}{b}}{\frac{c}{d}} = \frac{ad}{bc}$
    \end{enumerate}
\end{frame}


\begin{frame}
    \frametitle{Binomische Formeln Übungen}
    \begin{enumerate}
        \setlength\itemsep{1em}
        \item $\frac{a}{b} * (1 - \frac{c}{1-c})$
        \item $\frac{a}{\ln(1)}$
        \item $\frac{6p + 11}{2p + 4} - \frac{2p + 5}{p^2 +2p} -3$
    \end{enumerate}
\end{frame}

\begin{frame}
    \frametitle{Binomische Formeln Übungen Lösungen}
    \begin{enumerate}
        \setlength\itemsep{1em}
        \item $\frac{a}{b} * (1 - \frac{c}{1-c}) = \frac{ac}{b(1-c)}$
        \item $\frac{a}{\ln(1)}$ Nicht erlaubt, da durch 0 geteilt wird
        \item $\frac{6p + 11}{2p + 4} - \frac{2p + 5}{p^2 +2p} -3 = -\frac{5}{2p}$
    \end{enumerate}
\end{frame}

\begin{frame}
    \frametitle{Binomische Formeln}
    \begin{enumerate}
        \setlength\itemsep{1em}
        \item $(a+b)^2 = a^2 + 2ab + b^2$
        \item $(a-b)^2 = a^2 - 2ab + b^2$
        \item $(a+b)(a-b) = a^2 - b^2$
    \end{enumerate}
\end{frame}

\begin{frame}
    \frametitle{Binomische Formeln Übungen}
    \begin{enumerate}
        \setlength\itemsep{1em}
        \item $(2+r)^2 - (2-r)^2$
        \item $\frac{1}{2}a^2 - 4ab + 4b^2$
        \item $(0.5-y)^2 - (0.5 - y)*(0.5 + y)$
    \end{enumerate}
\end{frame}

\begin{frame}
    \frametitle{Binomische Formeln Übungen Lösungen}
    \begin{enumerate}
        \setlength\itemsep{1em}
        \item $(2+r)^2 - (2-r)^2 = 8r$
        \item $\frac{1}{2}a^2 - 4ab + 4b^2$ hier kann keine binomische Formel angewendet werden
        \item $(0.5-y)^2 - (0.5 - y)*(0.5 + y) = -xy+2y^2$
    \end{enumerate}
\end{frame}


\begin{frame}
    \frametitle{Potenzgesetze}
	\begin{itemize}
		\setlength\itemsep{1em}
		\item $a^m * a^n = a^{m+n}$
		\item $\frac{a^m}{a^n} = a^{m-n}$
		\item $a^n * b^n = (ab)^n$
		\item $\frac{a^n}{b^n} = \frac{a}{b}^n$
		\item $(a^m)^n = a^{m*n}$
		\item $a^0 = 1$
		\item $a^{-n} = \frac{1}{a^n}$
	\end{itemize}
\end{frame}

\begin{frame}
    \frametitle{Potenzgesetze Übungen}
	\begin{enumerate}
		\setlength\itemsep{1em}
		\item $\frac{1^3}{5^n}$
		\item $\frac{5^0- \sin(a^2) + \cos(a^2)}{\ln(e)}$
		\item $x^{-n} * x$
		\item $\frac{x^5 +1}{x^{m+2}} - \frac{2x^2-2}{x^m} + \frac{2-x}{x^{m-2}}$
	\end{enumerate}
\end{frame}

\begin{frame}
    \frametitle{Potenzgesetze Lösungen}
	\begin{enumerate}
		\setlength\itemsep{1em}
		\item $\frac{1^3}{5^n} = \frac{1}{5}^n$
		\item $\frac{5^0- \sin(a^2) + \cos(a^2)}{\ln(e)} = 1$
		\item $x^{-n} * x = x^{-n+1}$
		\item $\frac{1+2x^2}{x^{m+2}}$
	\end{enumerate}
\end{frame}
